\section{Minimize color switches with matrices}

The previous section provides a strategy to compute the smallest cost of a given path. It has been shown that an optimal strategy is to delay color switches as mush as possible. In this section we reuse this concept in order to find paths with a \textit{fixed} number of edges between two vertices, minimizing the number of color switches.

\subsection{\FW\ algorithm}
\label{sec:fwalgo}

Floyd \cite{floyd} and Warshall \cite{warshall}, in respectively 1959 and 1962, gave an implementation \cite{floydalgo} of an algorithm able to compute the shortest path between any pair of vertices of a directed weighted graph. The solution is found in polynomial time over the number of vertices of the graph.

In particular, let $\graphdef$ be a directed graph and a cost function $\weight : \aset \rightarrow \mathbb{N}$, such that for all pair of vertices $i, j$ of $\vset$, if there exists no arc going from $i$ to $j$ in $\aset$ then $\weight(i,j) = \infty$ and for each $\n \in \vset$, $\weight(\n, \n) = 0$ (\ie\ we are creating self loops on every vertex of the graph, that is we can stay in a vertex at no cost). Let $\adjmat$ be the $\vcard \times \vcard$ adjacency matrix of $\graph$ such that each cell $M[i][j]$ equals $\weight(i, j)$.

The goal of the algorithm is to build a new matrix $\matr$ whose cells contain the weight of the shortest path for every pair of vertices. This matrix is updated iteratively: at time $1$, we have $\matr^1 = \adjmat$: $N^0$ contains all the shortest path of length \textit{at most} $1$ between two vertices. This information should however be reworked because it can exist paths of smaller length made of more than one arc. Therefore, for every pair $i, j \in \vset \times \vset$, we seek if it exists a shortest path from $i$ to $j$ passing through a third vertex $k$ exists.

\begin{equation}
  \label{eq:cellc}
  \matr^k[i][j] = \min_{\n \in \vset} (\matr^{k-1}[i][\n] + \matr^{k-1}[\n][j])
\end{equation}

At the second iteration, we obtain $\matr^2$ which contains all the shortest paths of length \textit{at most} $2$ for every pair of vertices. Globally, the matrix should be updated $\vcard$ times since, except for negative cycles, every shortest path between two vertices will pass through every vertex \textit{at most} one time.

The overall time complexity of the \FW\ algorithm is $\bigo(\vcard^3)$, we need to make at most $\vcard$ update of a $n \times n$ matrix.

\begin{figure}
  % [!htb]
  \centering
  \begin{tikzpicture}[->,>=stealth',shorten >=1pt,auto,node distance=3cm, scale = 0.6,transform shape]

    \node[state] (A) {$\n_1$};
    \node[state] (C) [below of=A] {$\n_3$};
    \node[state] (D) [right of=C] {$\n_4$};
    \node[state] (B) [right of=A] {$\n_2$};

    \path
    (A) edge [right, bend left=20]  node {$3$} (C)
    (A) edge   node {$-4$} (B)
    (C) edge [right, bend left=20]  node {$2$} (A)
    (C) edge   node {$4$} (B)
    (D) edge   node {$1$} (C)
    (B) edge   node {$2$} (D);

  \end{tikzpicture}
  \caption{A directed weighted graph example}
  \label{fig:digraph}
\end{figure}

\begin{table}
  % [!htb]
  \footnotesize
  \newcommand\makematrix[9]{
    \neworrenewcommand{\ffoo}[7]{
      \begin{tabular}{c|cccc}
               & $\n_1$ & $\n_2$ & $\n_3$ & $\n_4$ \\ \hline
        $\n_1$ & $#1$   & $#2$   & $#3$   & $#4$   \\
        $\n_2$ & $#5$   & $#6$   & $#7$   & $#8$   \\
        $\n_3$ & $#9$   & $##1$  & $##2$  & $##3$  \\
        $\n_4$ & $##4$  & $##5$  & $##6$  & $##7$
      \end{tabular}
    }
    \ffoo
  }
  \begin{subtable}{.25\linewidth}
    \centering
    \makematrix
    {0}{-4}{3}{\infty}
    {\infty}{0}{\infty}{2}
    {2}{4}{0}{\infty}
    {\infty}{\infty}{1}{0}
    \caption{Iteration 1}
    \label{tbl:floydit1}
  \end{subtable}%
  \begin{subtable}{.25\linewidth}
    \centering
    \makematrix
    {0}{-4}{3}{-2}
    {\infty}{0}{\infty}{2}
    {2}{-2}{0}{0}
    {\infty}{\infty}{1}{0}
    \caption{Iteration 2}
  \end{subtable}%
  \begin{subtable}{.25\linewidth}
    \centering
    \makematrix
    {0}{-4}{3}{-2}
    {\infty}{0}{\infty}{2}
    {2}{-2}{0}{0}
    {3}{-1}{1}{0}
    \caption{Iteration 3}
  \end{subtable}%
  \begin{subtable}{.25\linewidth}
    \centering
    \makematrix
    {0}{-4}{3}{-2}
    {5}{0}{3}{2}
    {2}{-2}{0}{0}
    {3}{-1}{1}{0}
    \caption{Iteration 4}
  \end{subtable}

  \caption{Floyd-Wharshall execturion of \cref{fig:digraph}}
  \label{tbl:floydexec}
\end{table}

\begin{example}[\FW\ algorithm run]
  Let's take the directed graph represented in \cref{fig:digraph}. The corresponding matrix $\adjmat$ is indicated in \cref{tbl:floydit1}. At the iteration $4$, the distance from the vertex $\n_1$ to vertex $\n_3$ is updated to $-1$ since there is a shorter path going from $n_1$ to $\n_4$ and then from $\n_4$ to $\n_3$. Its overall cost is given by $c_{1,4} + c_{4,3} = -2 + 1 = -1$ which is less than the direct path $\n_1$ to $\n_3$.
\end{example}

\subsection{Paths of fixed length with minimum cost}

As explained in \cite{floydGeneric} and \cite{cpweb}, the \FW\ algorithm can be generalized in order to compute shortest paths on directed weighted graphs having a \textit{fixed} number of edges. This approach is called the \FW\ generalized algorithm and is based on the theory of semirings \cite{ullman}.

\paragraph{Semiring} A \textit{semiring}\cite{semiring} is a algebraic structure composed by a set $R$ and two binary operators $\oplus$ and $\otimes$. $(R, \oplus)$ forms a commutative monoid with an identity element $z$. $(R, \otimes)$ forms a monoid with an identity element called $e$. $\oplus$ is left and right distributive over $\oplus$ and $z$ absorbs $\otimes$. A semiring differs from a ring because the $\oplus$ does not need to have an inverse element for $r \in R$.

\paragraph{\FW\ generalized algortithm} Let $\adjmat$ be the adjacency matrix of a graph whose cells on the diagonal have infinity weights if there is no self-loop on the considered vertex. We say that $\matr^\len$ is the matrix where each cell $\matr[i][j]$ contains the cost of the shortest path from $i$ to $j$ with \textit{exactly} $k$ edges\footnote{Note that in \cref{sec:fwalgo} we spoke about path of \textit{at most} $k$ edges.}. The update function of this generalized approach differs from \cref{eq:cellc} since the cost of the cell $\matr[i][j]$ at time $\len$ will depend of the cost of the iteration $\matr^{k - 1}$ and the adjacency matrix $\adjmat$.

\begin{equation}
  \label{eq:fw_gen}
  \matr^k[i][j] = \min_{\n \in \vset} (\matr^{k-1}[i][\n] + \adjmat[\n][j])
\end{equation}

The time complexity of this computation is $\bigo(\vcard^3 \len)$, since to pass from $\matr^i$ to $\matr^{i+1}$ we must read $\vcard$ time the $\vcard \times \vcard$ matrix and globally the matrix is updated $\k$ times.

\paragraph{Link with semirings} It is possible to rewrite this equation in a more concise way using the definition of semiring. In fact, if, from \cref{eq:fw_gen}, the $\min$ operator is the $\oplus$ and the $+$ operator is the $\otimes$. We have that $\matr^k = \oplus (\matr^{k-1} \otimes \adjmat)$. We can further simplify the notation knowing that $\matr^1 = \adjmat$ and $\matr^k = \matr^{k-1} \odot \adjmat$. In fact $\matr^k = M^{\odot \len}$ and since $\min$ and $+$ are associative, we can improve the previous complexity using the binary exponentiation \cite{binexp} and get $\bigo(\vcard^3 \log \len)$.

% An important remark of this approach is that while finding path with fixed length $k$, we consent to pass more than one time through each vertices (\ie we can have non-simple path).

\subsection{Minimize color switches with matrices}
\label{sec:algo_matrix}

In this section we propose an adaptation of the generalized \FW\ algorithm in order to compute shortest paths of fixed length minimizing the number of color switches in oriented graphs. This adaptation wants to merge this procedure with the idea of delaying color switches proposed in \cref{sec:path_proc}.

The adjacency matrix $\adjmat$ is defined differently, since we do not have exact costs associated to arcs: the cost depends on the color assignation of two adjacent edges. In our implementation, $\adjmat[i][j]$ is replaced by the coloring function $\colf(\n_i, \n_j)$ with the particularity that $\colf(\n_i, \n_j) = \varnothing$ if there is no arc from i to j.

The cells of the $\matr^1$ matrix is a pair \textit{(w, cols)} where: \textit{w} is the cost of the path and \textit{cols} is the set of colors minimizing the number of color switches for the path going for each vertex $\n_i$ to $\n_j$.

Similarly to the matrix computation illustrated in the previous section, $M^\len[i][j]$ depends on the matrix at time $\len-1$ and $\colf$. For all $\n_i, \n_j \in \vset \times \vset$, $\matr^1_{ij} = \{w \gets 0 \textit{ if } \colf(\n_i, \n_j) \neq \varnothing \text{ otherwise } \infty;\; cols \gets \colf(\n_i, \n_j)\}$. The $\matr^{k+1}$ is computed by \cref{algo:nextM}.

\begin{algorithm}
  \nextfloat
  \caption{Compute $\matr^{\len + 1}$}
  \label{algo:nextM}

  \KwIn{$\matr^\len,\; \colf$, respectively, the matrix at time $\len$ and the coloring function}
  \KwOut{$\matr^{\len + 1}$ the matrix at time $\len + 1$}
  \def\res{\matr^{\len+1}}

  $n \gets$ the number of vertices of the graph\;

  \tcp{Matrix initialization}

  $\res \gets \text{ new } n \times n \text{ matrix }$\;
  $\forall i, j \in [0 .. n]^2: \; \res[i][j] \gets \{w \gets \infty;\; cols \gets \varnothing\}$\;

  \tcp{Procedure start}

  \For{$i = 1$ \KwTo $n$}{
    \For{$j = 1$ \KwTo $n$}{
      \For{$v = 1$ \KwTo $n$}{
        $\mathcal{I} \gets \matr^\len[i][v].cols \cap \colf(v, j)$\;
        $cost \gets \matr^\len[i][v].w + (\text{if } \mathcal{I} = \varnothing \text{ then } 1 \text{ else } 0)$\;
        $\mathcal{S} \gets (\text{if } \mathcal{I} = \varnothing \text{ then } \colf(v, j) \text{ else } \mathcal{I})$\;
        \uIf{$cost < \res[i][j].w$} {
          $\res[i][j] \gets \{w \gets cost;\; cols \gets \mathcal{S}\}$\;
        } \ElseIf {$cost = \res[i][j].w$}{
          $\res[i][j].cols \gets\; \mathcal{S} \cup \res[i][j].cols$\;
        }
      }
    }
  }
  \Return $\res$\;

\end{algorithm}

\paragraph{Analyze of \cref{algo:nextM}}
The first step of the algorithm initiates the matrix to return $\matr^{k+1}$. The cells of this matrix have an empty set of colors and an infinity cost. After this initialization, we loop over each pair of vertices $(i, j)$ and, as for the generalized version of the \FW\ algorithm, we look for minimal paths passing through each vertex $v \in \vset$. This distance is obtained wrt the result of the intersection $\mathcal{I}$ between the color set of $\matr^\len[i][v]$ and $\colf(v,j)$. If $\mathcal{I}$ is not empty then we are able to avoid a color switch and, therefore, the cost of the path from $i$ to $j$ passing through $v$ is the same as the cost of the path from $i$ to $v$. On the other hand, if the intersection is empty, the cost of the path will be $1$ more than the cost of the path from $i$ to $v$. $\mathcal{S}$ is the set of colors that can be associated to the arc $(v,j)$. It is equal to $\mathcal{I}$ if $\mathcal{I}$ is non-empty, otherwise, it will be affected to $\colf(v,j)$, since any color in $\colf(v,j)$ will force a color switch.

% Let $0 \leq p' \leq p \leq \vcard$. While looping over all the intermediate vertex $\n_p$, we can have three possible scenarios:
% \begin{itemize}
%   \item there exists a path passing through $\n_{p'}$ which is less than the path passing through $\n_p$, the path through $n_p$ can be ignored;
%   \item the computed cost is strictly less than all the the previous path passing through $\n_{p'}$, in this case the shortest path between $ij$ will take this new cost and its set of colors will $\mathcal{S}$;
%   \item finally, the cost passing through $\n_i$ equals a previous minimal one. In this case, the cost is not updated, but the colors we can give to the edge $ij$, to reduce the possibility of a switch, will be the union of the colors of the previous best affectations and $\mathcal{S}$.
%         \change{Add binary exponentiation}
% \end{itemize}
