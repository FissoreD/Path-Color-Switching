\section{Minimize Switches in Paths}

The goal of this section is to provide a greedy algorithm able to compute an optimal assignement $\affecto$ of a given path $\path$. The obtained result, will then be extended to general graphs using a matrix technique proposed in \cref{sec:algo_matrix} and the MDD strategy proposed in \cref{sec:algo_mdd}.

\subsection{Procedure}
\label{sec:path_proc}

Let $\path = \row{\n}{\len}$ be a path in $\graph$, the greedy strategy to find an optimal assignement $\affecto$ is to \textit{delay} a color switch as much as possible. The algorithm is decomposed in two main parts: the first (\textit{part. A}) assigns each arc $\e_i \in \path$ to a subset of colors chosen from $\colf(\e_i)$ and the second (\textit{part B}) returns the optimal assignement $\affecto$.

\paragraph{Part A.}
In this part of the procedure, we affect each arc of $\path$ to $\mathcal{L} = \row{\C}{\len}$, such that forall $1 \leq i \leq \len$, $\C_i$ is a subset of $\colf(\e_i)$. Firstly, $\C_1$ is exactly $\colf(\e_1)$. Next, forall $1 < i \leq \len$, the set of colors $\C_i$ attributed to the arc $\e_i$ will be iteratively set to the intersection between $\C_{i-1}$ and $\colf(\e_i)$ if the intersection is non-empty, otherwise $\C_i$ will be affected to $\colf(\e_i)$.

\paragraph{Part B.}
In this second part of the procedure, we make a unique color affectation from the list $\mathcal{L}$ returned by the \textit{part A}. This time, we read $\mathcal{L}$ from right to left. The last arc is assigned to a random color $\c$ chosen from $\C_\len$. Then forall $0 \leq i < \len$, the color of then $i^{th}$ arc is $\c_{i+1}$, if $\c_{i+1}$ belongs to the set $\C_i$, otherwise, we are facing a color switch, and, the arc $\e_i$ can be assigned to an arbitrary color $\c$ chosen from $\C_i$.

An implementation of this procedure, containing both part of the algorithm, can be found in \cref{algo:minpath}.

Here, we want to give a formal proof to show that the stated procedure returns an optimal assignement for any given path. This proof is decomposed in two parts, one for each subpart of the global algorithm. In the first part we show that the list $\mathcal{L}$ minimizes the number of color switches and in second part we show that the cost of the assignement returned by \textit{part B} is the same of the one returned by \textit{part A}.

\begin{proof}[\normalfont\textbf{Proof of \textit{part A}}]
	\def\solPartOne{\affect}
	Let $\solPartOne = \row{\C}{\len}$ be the solution returned by \textit{part A}, we prove, by induction on the length of the path, that $\solPartOne$ minimizes the number of color switches. \\
	By definition of the weight function, if the length $\len$ of the path is $1$, then $\path = (e_1)$ and we have $\weight(\solPartOne) = 0$ which is the optimal cost: any color chosen from $\colf(\e_1)$ will cause no color switch.\\
	In this inductive part of the proof, we suppose that $\solPartOne$ is an optimal solution for every path of length at least $\len$. We want to prove that the new affectation $\solPartOne'$ returned by the algorithm for a path of length $\len+1$ is still optimal. \\
	This proof can be done by a case-by-case analyze:

	\begin{itemize}
		\item if $\colf(\e_k) \cap \colf(\e_{k+1}) = \varnothing$ then we are forced to make a color switch between $\e_k$ and $\e_{k+1}$, since, the intersection of the colors of the two arcs is empty. In this particular scenario, the cost of the affectation returned for the path of length $\len + 1$ will be $\weight(\solPartOne') = \weight(\solPartOne) + 1$. Since, by the induction hypothesis, $\solPartOne$ is optimal, $\weight(\affect')$ remains optimal.
		\item otherwise, if $\colf(\e_k) \cap \colf(\e_{k+1}) \neq \varnothing$ we have two sub-cases to treat:
		      \begin{itemize}

			      \item if it exists a subset of colors $\C_{\len+1} \subseteq \colf(\e_{\len+1})$ which is included in $\C_\len$, we are able to avoid a color switch since we are able to attribute the same color to $\e_\len$ and $\e_{\len+1}$, therefore, the cost of the affectation $\solPartOne'$ of the new path of length $k+1$ equals $\weight(\solPartOne)$. Again, since the affectation $\solPartOne$ is optimal, and we do not increase the number of color switches then the new affectation $\solPartOne'$ is still optimal.

			      \item this final case is the most interesting to treat because the intersection between $\C_{k}$ and $\colf(\e_{\len+1})$ is empty, but, on the other hand, $\colf(\e_{\len}) \cap \colf(\e_{\len+1}) \neq \varnothing$. It means that the particular choice of colors associated to the arc $\e_\k$ is causing a color switch, even if it had been possible to make no color break between the $\len^{th}$ arc and the  $(\len+1)^{th}$ arc of $\path$. The cost of the affectation $\solPartOne'$ is therefore, $\weight(\solPartOne) + 1$.\\
			            Let's suppose, by means of contradiction, that there exists a better affectation $\solPartOne_{OPT}$. Without loss of generality, let's suppose that the intersection of the colors of the first $\len$ arcs of the path is not empty, \ie\ there exists at least one color shared by all the $\e_i$ ($0 \leq i \leq \len$) first arcs. The cost of this subpath is $0$ since all of the arcs can have the same color. If we want to add the new arc $\e_{\len + 1}$ to the path, without increasing the number of color switches, then it must exist at least one color belonging to $\bigcap\limits_{i = 1}^{\len + 1} \colf(i)$. However, this condition is not possible, otherwise the algorithm would have kept this subset of color as a valid option for every arc of the path, but, by hypothesis we have that $\C_\len$ and $\colf(\e_{\len + 1})$ is empty. A contradiction.
		      \end{itemize}
	\end{itemize}
	We can conclude that the number of color switches returned by the first part of the procedure is minimal, therefore, optimal.
\end{proof}

\begin{proof}[\normalfont\textbf{Proof of \textit{part B}}]
	In the previous proof, we have shown that the number of color switches returned by \textit{part A} is minimal. We only have to prove that the second part of the procedure returns an assignement with the same number of color switches.\\
	Let $\mathcal{L} = \row{\C}{\len}$ be the subset affectation returned by \textit{part A}. By construction of the \textit{part A}, for each set $\C_i$ of $\mathcal{L}$, either $\C_{i+1}$ is a subset of $\C_i$ or $\C_i \cap \C_{i+1} = \varnothing$.
	Starting from the last arc of the path, we can choose an arbitrary color $\c_\len \in \C_\len$ for the arc $\e_\len$. Then for the arc $\e_{\len-1}$, we choose the same color of $\e_\len$ if possible and repeat the same procedure until reaching the first arc of the path.\\
	We have, therefore, a color switch only when the intersection of $\C_i$ and $\C_{i-1}$ is empty.
\end{proof}


\subsection{Time Complexity}

We can analyze the time complexity of this procedure from the implementation proposed in \cref{algo:minpath}. We have two loops of size $k$ (the length of the path). Inside them we make intersection between sets of at most $| \C |$, knowing that the intersection between two sets of size $| \C |$ is $\bigo(| \C |)$. The final time complexity is therefore $\bigo(2 * k * | \C |) = \bigo(k * | \C |)$ which is an optimal time complexity wrt the input of the problem.

\subsection{An example run}
\label{sec:path_ex_run}

\begin{figure}[!htb]

	\centering
	\begin{tikzpicture}
		\def\v{0.25}
		\def\h{0.26}
		\def\sep{2.2}

		\coordinate (a) at (\sep*0,0);
		\coordinate (b) at (\sep*1,0);
		\coordinate (c) at (\sep*2,0);
		\coordinate (d) at (\sep*3,0);
		\coordinate (e) at (\sep*4,0);
		\coordinate (f) at (\sep*5,0);
		\coordinate (g) at (\sep*6,0);

		% Draw dots with name
		\foreach \x [count=\i] in {a,b,c,d,e,f,g} {
				\node at (\x) [circle,fill,inner sep=1.5pt]{};
				\node[above=1pt of {\x},fill opacity=0,text=black, text opacity=1] {$\n_\i$};
			}

		\def\lis{
			{\eAcols}/3/a/b,
			{\eBcols}/3/b/c,
			{\eCcols}/2/c/d,
			{\eDcols}/1/d/e,
			{\eEcols}/3/e/f,
			{\eFcols}/2/f/g}
		\foreach[expand list] \q/\l/\x/\y [count=\i] in \lis{
			\draw [opacity=0] ($(\x)+(\h,-1*\v)$) -- node[below,black,opacity=1] {$\e_\i$} ($(\y)+(-\h,-1*\v)$);
			\pgfmathsetmacro{\sp}{\l/2+0.5};

			\foreach \c [count=\i] in \q {
				\pgfmathsetmacro{\coeff}{(\i-\sp)*\v};
				\draw [line width=2pt, \c, -to]  ($(\x)+(\h, \coeff)$) -- ($(\y)+(-\h, \coeff)$);
			}
		}

	\end{tikzpicture}

	\caption{A path example}
	\label{fig:path_example}

\end{figure}

Let's take \cref{fig:path_example}, where $\path = \row{\e}{6}$ and $\colf$ such that
\begin{align*}
	\colf(\path) = ( & \{\eAcols\}, \{\eBcols\},    \\
	                 & \{\eCcols\}, \{\eDcols\},    \\
	                 & \{\eEcols\}, \{\eFcols\}   )
\end{align*}

Here we give a solution of how the procedure proposed in \cref{sec:path_proc} would solve it. The list $\mathcal{L}$ returned by \textit{Part A} will be
\begin{align*}
	\mathcal{L} = ( & \{\eAcols\}, \{\colB\},      \\
	                & \{\colB\}, \{\colC\},        \\
	                & \{\eEcols\}, \{\eFcols\}   )
\end{align*}

Then the second part of the algorithm would return an optimal solution which is, in this case, $\affect = (\colB, \colB, \colB, \colC, \colD, \colD)$, with $\weight(\affect) = 2$.

One can note that there can exist other optimal solutions, from \cref{fig:path_example} we can choose $\affect_2 = (\colA, \colC, \colC, \colC, \colB, \colB)$, but none of them will have a cost smaller than $\weight(H)$.

\subsection{Extension on cycles}

A cycle is a path whose starting end ending nodes coincide.
In this situation, the previous algorithm is no more effective, since we need to keep into account the potential color switch between the first and the last arcs.
We can, however, easily modify the procedure proposed in \cref{sec:path_proc}, to compute optimal assignement on cycles.

Let's take the path of \cref{fig:path_example} and imagine that nodes $n_1$ and $n_7$ coincide.
The affectation $\affect$ of \cref{sec:path_ex_run} is no more optimal since $\weight(\affect) = 3$, whereas the cost of the affectation $\affect' = (\colB, \colB, \colB, \colC, \colB, \colB)$ is $2$.

In order to take into account this situation, we have to look at the intersection between the first and the last set of colors returned by \textit{part A}. If the intersection between the sets of $\C_1$ and $\C_\len$ is not empty, we set them into their intersection.

Concretely, let's consider the example in \cref{fig:path_example}, we intersect $\C_1$ with $\C_6$. Since this intersection $\mathcal{I}$ is non-empty, then $\C_1 \gets \mathcal{I}$ and $\C_6 \gets \mathcal{I}$. The resulting affectation will be exactly $H'$ which has the optimal cost.

