\section{An implementation of the stated procedures}

The previous algorithms have been implemented in \textit{OCaml} and here we want to provide a sketch of the main data structures used to fulfill the requirements.

\paragraph{MySet.} A useful data structure extending the classical \textit{Set} module of \textit{OCaml}. In particular, when we start to build a path, in order to maintain the \textit{standard} update function over colors for each couple on adjacent nodes of a path, we need to represent the ``\textit{Full}'' set. The classical operation over sets have been overrode if needed, so that, for example, the intersection of a set $\mathcal{S}$ and the \textit{Full} set gives $\mathcal{S}$ and their union gives \textit{Full}. The main advantage of sets in \textit{OCaml} is that they are an immutable data structure. In fact, every binary operation over a set does not modify the current set, but it builds rapidly a fresh copy with the wanted content. This is useful in our \mdd\ implementation, for example, since the \textit{colors} of each state should contain the intersection of the colors of the father and the colors of the current arc. We use the \textit{Full} set to represent the set $\C$ of all the colors inside the graph.

\paragraph{The coloring function $\colf$.} Working with the implementation of the proposed algorithms, we remarked that a clear type for the color function would have improved a lot the clarity of the code.

\begin{minted}{ocaml}
  type colorFunction = {
    is_sym : bool;
    tbl : (int * int, ColorSet.t) Hashtbl.t;
    get_col : int * int -> ColorSet.t;
  }

  let get tbl (v1, v2) =
    Option.value ~default:ColorSet.empty (Hashtbl.find_opt tbl (v1, v2))

  let init ?(is_sym = false) () =
    let tbl = Hashtbl.create 2048 in
    { is_sym; tbl; get_col = get tbl }

  let add { tbl; is_sym; _ } v1 v2 col =
    Hashtbl.replace tbl (v1, v2) col;
    if is_sym then Hashtbl.replace tbl (v2, v1) col
\end{minted}

The \textit{is\_sym} field is used internally to know if the graph is undirected: in this case every edge $(a, b)$ of the undirected graph is transformed to a couple of directed arcs $(a,b)$ and $(b,a)$. The \textit{tbl} field is an hash-table where to each couple of nodes we associate the set of colors of the arc they represent. The \textit{get\_col} is a function taking a couple of node which gives back either the set of colors of the arc if the arc between the two nodes exists otherwise the empty set. Finally, the \textit{color\_function} can be instantiated through the \textit{init} function and the \textit{add} function allows to add new arcs of the graph to it.

\paragraph{A state of a \mdd.} A state of a \mdd\ is represented by the following module-type:

\begin{minted}{ocaml}
  module type State = sig
    type t2 = { color : ColorSet.t; w : int }
    type t = { name : int; mutable father : t list; content : t2 }

    val compareForUnion : t -> t -> action
    val mergeAction : t -> t -> t
  end
\end{minted}

They have a type \texttt{t} (following the Ocaml conventions) containing the name of the current node, the list of father\footnote{The list is empty if we are implementing the classic algorithm without the \alldiff\ variant} and a content which is made of a set of colors and the cost of the current state \texttt{w}.

\paragraph{The \mdd\ functor.} The final important data structure is the \mdd\ functor. This functor takes as parameter the implementation of a \textit{State} of a \mdd.

\begin{minted}{ocaml}
  module Make : functor (T : State) ->
    sig
      ...
      val initiate : ColorFunction.colorFunction -> int -> graph
      val make_iteration : graph -> unit
      val run : graph -> int -> unit
    end
\end{minted}

This functor signature implements some useful functions such as the \textit{initiate} function, the \textit{make\_iteration} function which add the new layer to the \mdd, the \textit{run} function which finds all the paths of a given length.
