\section{An implementation of the stated procedures}

The previous algorithms have been implemented in \textit{OCaml}, following the procedures provided in the previous sections, and here we want to provide a sketch of the main data structures used to fulfill the requirements.

\paragraph{MySet.} A useful data structure extending the classical \textit{Set} module of \textit{OCaml}. In particular, when we start to build a path, in order to maintain the \textit{standard} update function over colors for each couple on adjacent nodes of a path, we need to represent the ``\textit{Full}'' set (\ie\ the complement of $\varnothing$). \textit{MySet} adds this specifications. The classical operation over sets have been overrode if needed, so that, for example, the intersection of a set $\mathcal{S}$ and the \textit{Full} set gives $\mathcal{S}$ and their union gives \textit{Full}. The main advantage of sets in \textit{OCaml} is that they are an immutable data structure. In fact, every binary operation over a set does not modify the current set, but it builds rapidly a fresh copy with the wanted content. This is very useful in our \mdd\ implementation, for example, since the \textit{parents} filed of each child of a state should contain the intersection of the colors of the father and the colors of the current arc.

\paragraph{The color\_function type.} Working with the implementation of the proposed algorithms, we remarked that a clear type for the color function would have improved a lot the clarity of the code.

\begin{minted}{ocaml}
  type color_function = {
    is_sym : bool;
    tbl : (int * int, ColorSet.t) Hashtbl.t;
    get_col : int * int -> ColorSet.t;
  }

  let get tbl (v1, v2) =
    Option.value ~default:ColorSet.empty (Hashtbl.find_opt tbl (v1, v2))

  let init ?(is_sym = false) () =
    let tbl = Hashtbl.create 2048 in
    { is_sym; tbl; get_col = get tbl }

  let add { tbl; is_sym; _ } v1 v2 col =
    Hashtbl.replace tbl (v1, v2) col;
    if is_sym then Hashtbl.replace tbl (v2, v1) col
\end{minted}

The \textit{is\_sym} field is used internally to know if the graph is not directed: in this case every edge $\langle a, b \rangle$ of the undirected graph is transformed to a couple of directed arcs $(a,b)$ and $(b,a)$. The \textit{tbl} field is an hash-table where to a couple of nodes we associate the set of colors of the arc they represent. The \textit{get\_col} which is a function taking a couple of node which gives back either the set of colors of the arc if the arc between the two nodes exists otherwise the empty set. Finally, the \textit{color\_function} can be instantiated through the \textit{init} function and the \textit{add} function allows to add new arcs of the graph to it.

In fact, the \textit{color\_function} record can be seen as an classical \textit{Java} object in the light \textit{OCaml} functional style.

\paragraph{A state of a MDD.} The states of a MDD are implemented with a functor\cite{ocamlfunctor} parametrized by a module implementing the \textit{merge} operation, since we have to know if, having two states, we have to merge them, to keep the first or the second state.

\begin{minted}{ocaml}
  type action = MERGE | REPLACE | IGNORE

  module type OrderedType = sig
    type t
  
    val merge : t -> t -> action
    val compare : t -> t -> int
  end
  
  module Make (Ord : OrderedType) = struct
    module MySet = MySet.Make (Ord)
    include MySet
  end
\end{minted}

This functor is particularly useful if we want to implement the classic algorithm, or the \alldiff\ variant, in both case, in every case, we will only need to respectively implement the \textit{merge} function.

Moreover, we can verify if two states are compatible, \ie\ ready to be merged, through the \textit{compare} function. In fact, a layer of a \mdd\ is a set of states. We know if two state are to be merged if the \textit{compare} method returns $0$.

\paragraph{The mdd\_tree record.} The last but not the least, important data structure is the \textit{mdd\_tree}, the heart of the \mdd\ procedure implementation.

\begin{minted}{ocaml}
  type 'a mdd_tree = {
    node : int list;
    state : 'a;
    mutable children : (int, 'a mdd_tree) Hashtbl.t;
    mutable father : 'a mdd_tree list;
  }
\end{minted}

A \textit{mdd\_tree} is a recursive data structure made of a state, the root of the current sub \mdd, a hash-table of children a list of fathers of the current state. Thanks to this \dots\change{Rename mdd\_tree to decorated state and add mdd\_layer}

\paragraph{Others}\change{Matrix mathod, adjacency matrix}