\section{A benchmark of the \textit{MDD} implementation}

We have been provided, by Spotify, a graph representing a sample of the problem this company is dealing with. This graph is on the form of a \textit{json} where a list of nodes are associated to a set of colors and a list of pairs $(\e_1, \e_2)$ representing the arcs of the graph. The problem is on the form as the one depicted in \cref{fig:col-graph}

The \mdd\ version of the algorithm has been tested on this graph in order to obtain some of statistics. In particular we have tested the number of solutions and the time taken for a given path length. The source of the path taken into account, in order to build the results, is the node $1$. The are no big differences of performance for other nodes taken as root of the \mdd.

\buildCouplePlot{Number of paths}{paths-number}{1}{2}{number_of_paths_from_1}{Number of paths}

\buildCouplePlot{Seconds}{time}{3}{4}{time_classic}{Time taken to compute paths}

\paragraph{Number of solutions}
In \cref{img:plots_1} and \cref{img:plots_2}, we can see that the number of solutions computed by respectively the classic algorithm and the algorithm with the \alldiff\ constraint give a curve with an exponential growth. However, we can also remark that the introduction of the \alldiff\ constraint reduces drastically the number of solutions, in particular, for a path with $10$ edges, there are about $9 \times 10^5$ solutions in the classic version against the about $3 \times 10^4$ of the \alldiff\ version. This difference is justified, as said in the previous sections, by the fact that the \alldiff\ constraints more the domain of the variables.

\paragraph{Time comparison} Another statistic we can analyze from the given input is the time taken to find the solutions. The time for the classic version of the algorithm is linear wrt the length of the path. On the other hand, the \alldiff\ version growth is exponential confirming the complexity given in the previous sections.

\begin{table}
  \centering
  \begin{subtable}{.47\linewidth}
    \centering
    \begin{tabular}{ccc}
      \textit{length} & \textit{min\_cost} & \textit{max\_cost} \\
      \hline
      $1$             & $0$                & $0$                \\
      $2$             & $0$                & $0$                \\
      $3$             & $0$                & $0$                \\
      $4$             & $0$                & $2$                \\
      $5$             & $0$                & $3$                \\
      $50$            & $0$                & $1$                \\
      $1000$          & $0$                & $1$                \\
    \end{tabular}
    \caption{Classic algorithm}
    \label{tbl:sol_stats1}
  \end{subtable}
  \quad
  \begin{subtable}{.47\linewidth}
    \centering
    \begin{tabular}{ccc}
      \textit{length} & \textit{min\_cost} & \textit{max\_cost} \\
      \hline
      $1$             & $0$                & $0$                \\
      $2$             & $0$                & $0$                \\
      $3$             & $0$                & $1$                \\
      $4$             & $0$                & $2$                \\
      $5$             & $0$                & $3$                \\
      $6$             & $0$                & $4$                \\
      $7$             & $0$                & $5$                \\
    \end{tabular}
    \caption{\alldiff\ version}
    \label{tbl:sol_stats2}
  \end{subtable}

  \caption[short]{Number of solution stats}
  \label{tbl:sol_stats}
\end{table}

\paragraph{Cost of the paths for a given length}
In this paragraph we want to give some results about the cost of the paths calculated through the \cref{eq:costf}. We can deduce some characteristics of the graph and the behavior of the algorithm from \cref{tbl:sol_stats}. This table represent the cost of the shortest-paths costs starting from the node $1$ of $\graph$. In particular let's take into account \cref{tbl:sol_stats1} which shows the cost of the paths for the classic version of the algorithm, since we are dealing with \mdd s, we can see that there always exists a node in the graph with the wanting distance with cost zero. On the other hand, among all the nodes, the maximum path cost we can obtain reach a pick for a path length of $5$ and then this cost fall again to $1$. This means that in the graph there should exist a lot of small cycles having cost zero that are taken in order to obtain the wanted path length and which, at the same time, minimize the total cost.

The \cref{tbl:sol_stats2}, shows the costs of the shortest paths starting from node $1$ using the \alldiff\ constraint. In this case we are not able to compute very long path due to the exponential complexity, but we are able to see that there exists some nodes on the graph at the wanting distance from node $1$ with cost zero. On the other hand, due to the \alldiff\ constraint we see that the maximum cost increase more than the classic algorithm, since we are unable to take an already explored node which can potentially reduce the cost of the path.